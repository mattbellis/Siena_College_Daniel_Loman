%% ****** Start of file apstemplate.tex ****** %
%%
%%
%%   This file is part of the APS files in the REVTeX 4 distribution.
%%   Version 4.1r of REVTeX, August 2010
%%
%%
%%   Copyright (c) 2001, 2009, 2010 The American Physical Society.
%%
%%   See the REVTeX 4 README file for restrictions and more information.
%%
%
% This is a template for producing manuscripts for use with REVTEX 4.0
% Copy this file to another name and then work on that file.
% That way, you always have this original template file to use.
%
% Group addresses by affiliation; use superscriptaddress for long
% author lists, or if there are many overlapping affiliations.
% For Phys. Rev. appearance, change preprint to twocolumn.
% Choose pra, prb, prc, prd, pre, prl, prstab, prstper, or rmp for journal
%  Add 'draft' option to mark overfull boxes with black boxes
%  Add 'showpacs' option to make PACS codes appear
%  Add 'showkeys' option to make keywords appear
%\documentclass[aps,prl,preprint,groupedaddress]{revtex4-1}
%\documentclass[aps,prl,preprint,superscriptaddress]{revtex4-1}
\documentclass[aps,prl,reprint,groupedaddress]{revtex4-1}

\usepackage{lipsum}
\usepackage{color}

% You should use BibTeX and apsrev.bst for references
% Choosing a journal automatically selects the correct APS
% BibTeX style file (bst file), so only uncomment the line
% below if necessary.
%\bibliographystyle{apsrev4-1}

\begin{document}

% Use the \preprint command to place your local institutional report
% number in the upper righthand corner of the title page in preprint mode.
% Multiple \preprint commands are allowed.
% Use the 'preprintnumbers' class option to override journal defaults
% to display numbers if necessary
%\preprint{}

%Title of paper
\title{Correlation of team statistics with performance in the NFL}

% repeat the \author .. \affiliation  etc. as needed
% \email, \thanks, \homepage, \altaffiliation all apply to the current
% author. Explanatory text should go in the []'s, actual e-mail
% address or url should go in the {}'s for \email and \homepage.
% Please use the appropriate macro foreach each type of information

% \affiliation command applies to all authors since the last
% \affiliation command. The \affiliation command should follow the
% other information
% \affiliation can be followed by \email, \homepage, \thanks as well.
\author{Daniel Loman}
\email[]{dh08loma@siena.edu}
\affiliation{Siena College}
\author{Matthew Bellis}
\email[]{mbellis@siena.edu}
\affiliation{Siena College}

%Collaboration name if desired (requires use of superscriptaddress
%option in \documentclass). \noaffiliation is required (may also be
%used with the \author command).
%\collaboration can be followed by \email, \homepage, \thanks as well.
%\collaboration{}
%\noaffiliation

\date{\today}

\begin{abstract}
abstract
\end{abstract}

% insert suggested PACS numbers in braces on next line
\pacs{}
% insert suggested keywords - APS authors don't need to do this
%\keywords{}

%\maketitle must follow title, authors, abstract, \pacs, and \keywords
\maketitle

%%%%%%%%%%%%%%%%%%%%%%%%%%%%%%%%%%%%%%%%%%%%%%%%%%%%%%%%%%%%%%%%%%%%%%%%%%%%%%%%
% body of paper here - Use proper section commands
% References should be done using the \cite, \ref, and \label commands
\section{Introduction}
	The NFL is a constantly evolving league. Since 1970 passing and rushing statistics have succomb to change from both rule changes and incoming trends. The following graph shows pass and rush yard data from 1970 to 2012.

%\begin{figure}[p]
%	\includegraphics{}
%\end{figure}

	From the graph you cannot dispute that the league has become a passing league. Before the 1978 season rules were implemented to improve both the passing attack and player safety. Since then the league has been subject to more subtle rule changes and passing trends, like the spread offense. Due to these causes NFL passing stats have skyrocketed in recent years while leaving the running game in the dust.
	Because the game is ruled by the passing offense, there exists the stigma that a good team must possess an elite quarterback to succeed in the NFL. The phrase "defense wins championships" is uttered far less and the running game has been extremely devalued. I decided to take a look correlations between passing yards, rushing yards, defensive passing yards, defensive rushing yards, and win percentage.

\section{Correlation Coefficients}
	The correlation coefficient is measure of linear correlation between two variables, and is given by equation (1).

\begin{equation}
r=\frac{\sum_{i=1}^n \left(X_i-\bar{X} \right) \left(Y_i-\bar{Y} \right)}{\sqrt{\sum_{i=1}^n \left(X_i-\bar{X} \right)^2} \sqrt{\sum_{i=1}^n \left(Y_i-\bar{Y} \right)^2}}
\end{equation}

It ranges between -1 and 1, with 1 representing a perfect correlation, 0 representing no correlation, and -1 representing a perfect inverse correlation.

\section{Estimates of the uncertainties}
	Of course, the correlation coefficient itself tells us nothing without the uncertainty. The bootstrap method is a computational method that uses resampling to find the uncertainty of the correlation coefficient, and was introduced in 1979 by Dr. Bradley Efron with his paper Bootstrap method: Another Look at the Jackknife. The bootstrap method involves creating several new data pairs the same size as the original data pair by randomly selecting elements from the original data pair. Each new data pair has a new correlation coefficient, which can be put into a separate array containing all the new correlation coefficients. This array will distribute close to a normal gaussian curve, and should peak around the correlation of the original pair. A predetermined range of the new correlation coefficients represents the uncertainty for the original. In this exercise I used an uncertainty ranges of 1 standard deviation and 95% for N=1000 correlation coefficients of randomly generated data pairs.

\section{Results}
The correlation coefficients.

\section{Interpretation of results}
Can we draw any conclusions?

\section{Conclusions}
Summary of the analysis.

%%%%%%%%%%%%%%%%%%%%%%%%%%%%%%%%%%%%%%%%%%%%%%%%%%%%%%%%%%%%%%%%%%%%%%%%%%%%%%%%

% Create the reference section using BibTeX:
\bibliography{paperbib}

\end{document}
%
% ****** End of file apstemplate.tex ******

